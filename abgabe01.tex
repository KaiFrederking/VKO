\documentclass[12pt,a4paper]{scrartcl}
\usepackage[utf8]{inputenc}
\usepackage[ngerman]{babel}
\usepackage{amsmath}
\usepackage{amsfonts}
\usepackage{amssymb}
\usepackage{graphicx}
\usepackage{siunitx}
\usepackage{esvect}
\usepackage{multirow}%Tabellenzellen verbinden
\usepackage{tabulary}%tolle Tabellen
\usepackage{enumitem}%Aufzählungen schöner
\usepackage{dsfont}%Zahlenmengensymbole
\usepackage{pdfpages}
\usepackage[left=1.5cm,right=1.5cm,top=2cm,bottom=3cm]{geometry}
\usepackage{hyperref}
\author{\large
\begin{tabular}{llr}
Malte Hamann & 6318952 &1hamann@informatik.uni-hamburg.de
\\ Evelyn Fischer & 0000000 & 3fischer@informatik.uni-hamburg.de
\\ Marco Jendryczko & 0000000 & 5jendryc@informatik.uni-hamburg.de
\\ Kai Frederking & 0000000 & 1frederk@informatik.uni-hamburg.de
\end{tabular}
}
\title{Übungen Vertiefung Kombinatorische Optimierung Sommersemester 2016\\\vspace{\baselineskip}\large Gruppe 2, Mo 10-12, Abgabe Blatt 01 \\Übungsleiter: \url{daniel.weissauer@uni-hamburg.de}}
\date{\today}

\newcommand{\prob}[1]{\vspace{.5\baselineskip}\begin{addmargin}[15pt]{0pt}\textbf{#1}\end{addmargin}}
\newcommand{\ein}[1]{\vspace{.5\baselineskip}\begin{addmargin}[15pt]{0pt}\textbf{Eingabe: }#1\end{addmargin}}
\newcommand{\fra}[1]{\vspace{.5\baselineskip}\begin{addmargin}[15pt]{0pt}\textbf{Frage: }#1\end{addmargin}}
\newcommand{\loesung}[1]{\vspace{.5\baselineskip}\begin{addmargin}[0pt]{0pt}\textbf{Lösung: }#1\end{addmargin}}

\newcommand{\pr}{$\leq_p$ }%polynomiell reduzierbar


\begin{document}
\maketitle
\vspace{-\baselineskip}
\hrule
\vspace{\baselineskip}
\textbf{Hausaufgaben zum 18. April 2016}
\begin{enumerate}
	\item
	\begin{enumerate}
	\item Als bekannt setzen wir voraus, dass 3D-Matching ein NP-vollständiges Problem ist. Entsprechend zu 3D-MATCHING definiert man $k$D-MATCHING für $k \geq 2$. Formulieren Sie das Problem 4D-MATCHING und zeigen Sie, dass 4D-MATCHING ein NP-vollständiges Problem ist.
	
	\loesung{hier dann die Antwort einfügen}
	
	\item Für jedes feste $k \geq 2$ definieren wir das Problem $k$-CLIQUE wie folgt:
	\prob{k-CLIQUE}
	\ein{Ein Graph $G = (V,E)$.}
	\fra{Enthält $G$ einen vollständigen Graphen mit $k$ Knoten?}
	Für welche $k\geq 2$ ist $k$-CLIQUE ein NP-vollständiges Problem?
	
	\textbf{Hinweis:} Zur Lösung von Aufgabe 1 reicht es natürlich nicht, Antworten ohne Begründung zu geben; es kommt darauf an, Antworten zu geben \emph{und deren Richtigkeit nachzuweisen.}
	
	\loesung{hier dann die Antwort einfügen}
	\end{enumerate}

\item Gegeben sei eine Menge $A = \{a_1,\ldots,a_n\}$ sowie eine Kollektion $B_1,\ldots,B_m$ von Teilmengen von $A$. Eine Menge $H \subseteq A$ wird \emph{Hitting Set} für $B_1,\ldots,B_m$ genannt, falls $H \cap B_i \neq \emptyset$ für alle $i \in \{1,\ldots,m\}$ gilt. Wir betrachten das folgende Entscheidungsproblem:

	\prob{HITTING SET}
	\ein{Eine Menge $A = \{a_1,\ldots,a_n\},$ Teilmengen $B_1,\ldots,B_m$ von $A$ sowie eine Schranke $k \in \mathds{Z}, k \geq 1$.}
	\fra{Gibt es ein Hitting Set $H \subseteq A$ für $B_1,\ldots,B_m$, für das $|H| \leq k$ gilt?}
	\begin{enumerate}
	\item Beweisen Sie, dass HITTING SET ein NP-vollständiges Problem ist, indem Sie erstens nachweisen, dass HITTING SET in NP liegt, und zweitens 3-SAT zur Reduktion heranziehen.
	
	\loesung{hier dann die Antwort einfügen}
	
	\item Um die NP-Vollständigkeit von HITTING SET nachzuweisen, muss man nicht unbedingt 3-SAT verwenden. Fällt Ihnen eine andere (möglichst einfache) Reduktion eines NP-vollständigen Problems auf HITTING SET ein?
	
	\loesung{hier dann die Antwort einfügen}
	\end{enumerate}

\end{enumerate}

\end{document}